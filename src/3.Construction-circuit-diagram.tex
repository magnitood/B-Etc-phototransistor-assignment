\section{Circuit Diagram}
\begin{figure}[H]
    \centering
    \begin{circuitikz}[american] 
    \draw
    (0,0) node[npn, photo] (ps){};
    \node[anchor=west, font=\footnotesize] at (ps.collector) {\textit{C}};
    \node[anchor=north, font=\footnotesize] at (ps.base) {\textit{B}};
    \node[anchor=west, font=\footnotesize] at (ps.emitter) {\textit{E}};
    \end{circuitikz}
    \caption{NPN Phototransistor}
    \label{fig:symbol}
\end{figure}

\noindent Fig. \ref{fig:symbol} is the symbol for an npn phototransistor. Here the \(B\) represents base,
\(C\) represents collector and \(E\) represents emitter. The base terminal is the terminal where
light falls and generates a small current.