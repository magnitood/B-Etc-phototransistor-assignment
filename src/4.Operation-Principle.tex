\section{Operation Principle}
The operation of a phototransistor can be explained in four steps:
\begin{enumerate}
    \item \textbf{Light generation of electron-hole pairs:} When light strikes the base region of the phototransistor, it generates electron-hole pairs.
    The energy of the light photons must be greater than the bandgap energy of the semiconductor material in order to generate electron-hole pairs.
    \item \textbf{Diffusion of charge carriers:} The electron-hole pairs diffuse into the emitter-base junction.
    The electrons are attracted to the positive charge on the base, and the holes are attracted to the negative charge on the emitter.
    \item \textbf{Separation of charge carriers:} The electric field across the emitter-base junction separates the electron-hole pairs.
    The electrons are injected into the emitter, and the holes are collected by the collector.
    \item \textbf{Current flow:} The injected electrons flow from the emitter to the collector, and the collected holes flow from the base to the collector.
    This results in a small current flowing between the emitter and collector.
\end{enumerate}

\noindent The current gain of a phototransistor is determined by the following factors:

\begin{itemize}
    \item The efficiency of light generation of electron-hole pairs
    \item The diffusion length of the charge carriers
    \item The electric field across the emitter-base junction
    \item The current gain of the transistor itself
\end{itemize}